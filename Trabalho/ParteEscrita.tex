% Modelo Trabalho ABNT %

\documentclass[12pt,a4paper,oneside]{abntex2}
\usepackage[left=3cm,top=3cm,right=2cm,bottom=2cm]{geometry}
\usepackage[brazil]{babel} %lingua do texto
\usepackage[utf8]{inputenc} %permite caracteres como acentos e cedilhas
\usepackage{times} %usa a fonte times 
\setlength{\parindent}{1.25cm} %identação do parágrafo
\usepackage{indentfirst} % identa primeiro parágrafo % 
\usepackage{graphicx} %incluir figuras
\usepackage{lastpage} %para obter o número da última página
\pagenumbering{arabic} %Numeração de página

\author{Bernardo, Luiz Henrique Gariglio dos Santos, Naiara Barcelos Silveira}
\title{PROJETO MISTURADOR DE TINTA} 
\date{2023}
\instituicao{UNIVERSIDADE FEDERAL DE MINAS GERAIS\\Escola de Engenharia}
\local{Belo Horizonte}
\preambulo{Descrição do desenvolvimento do projeto de um misturador e tinta para a disciplina de Laboratório de Sistemas Digitais.}
\orientador{ Professor: Hermes Aguiar Magalhães}

\renewcommand{\imprimircapa}{
	\begin{capa}
		\begin{center}
            \begin{figure}[h]
				\centering
				\includegraphics[width=0.4\textwidth]{principal_completa_ufmg}
				\label{fig:principalcompletaufmg}
			\end{figure}
            \vspace{4cm}
    		{\normalsize\imprimirautor}\\[4cm]
    		{\normalsize\textbf{\imprimirtitulo}}\\
    		\vfill
    		{\normalsize\imprimirlocal}\\
    		{\normalsize\imprimirdata}
        \end{center}		
	\end{capa}
}

\newcommand{\folhaderostonova}{
	\begin{center}
        {\normalsize \imprimirautor } \\[5cm]
    	{\normalsize \textbf{\imprimirtitulo}} \\[4.5cm]
    	\hspace{.45 \textwidth} % posicionando a minipage
    	\begin{minipage}{.5\textwidth}
    		\imprimirpreambulo \\ \\
    		\imprimirorientadorRotulo \imprimirorientador \\ \\
    		\imprimircoorientadorRotulo \imprimircoorientador
    	\end{minipage}
    	\center
    	\vfill
    	{\normalsize \imprimirlocal } \\
    	{\normalsize \imprimirdata}
    \end{center}
}

%Corrigir títulos
\addto \captionsbrazil {\renewcommand{\listfigurename}{\normalsize\textbf{LISTA DE ILUSTRAÇÕES}}}
\addto \captionsbrazil {\renewcommand{\listtablename}{\normalsize\textbf{LISTA DE TABELAS}}}
\renewcommand{\listadesiglasname}{\normalsize\textbf{LISTA DE ABREVIATURAS E SIGLAS}}
\addto \captionsbrazil {\renewcommand{\contentsname}{\normalsize\textbf{SUMÁRIO}}}

\renewcommand{\ABNTEXchapterfontsize}{\normalfont\normalsize} 
\renewcommand{\cftchapterfont}{\normalfont}

\renewcommand{\ABNTEXsectionfontsize}{\normalsize\normalfont}
\renewcommand{\cftsectionfont}{\normalfont} 

\renewcommand{\ABNTEXsubsectionfontsize}{\normalfont\normalsize}
\renewcommand{\cftsubsectionfont}{\normalfont} 

\renewcommand{\ABNTEXsubsubsectionfontsize}{\normalfont\normalsize}
\renewcommand{\cftsubsubsectionfont}{\normalfont} 

\begin{document}
	\imprimircapa
	\folhaderostonova
	\newpage
	\listoffigures*
	\newpage
	\listoftables*
	\tableofcontents* \thispagestyle{empty} % Retira a numeração da página
	\mainmatter % Inicia a numeração das páginas
	
	\chapter{INTRODUÇÃO}
        O intuito deste trabalho é viabilizar uma máquina de mistura de tintas a partir dos conhecimentos construídos durante a disciplina de Laboratório de Sistemas Digitais. A ideia é que o usuário possa entrar com o código CMYK (Cyan - ciano, Magenta, Yellow - amarelo, Key - preto) na máquina e, a partir disso, receba uma tinta com a cor correspondente deste código. A simulação do modelo será feita em laboratório a partir do Kit Altera DE2, com um chip da família Cyclone II. Todo o código será feito em VHDL e as partes desenvolvidas da máquina serão dispostas nesse documento. \par
        Para essa construção, contamos com um diagrama de máquina de estados finitos de alto nível, um desenho do caminho de dados e os códigos. O resultado esperado é possibilitar uma simulação de como funcionaria uma máquina desse tipo, a partir de botões e leds disponibilizados no kit Altera DE2.
        
	\chapter{PROJETO}
        \section{Funcionamento Desejado}
            A máquina deverá receber do usuário um código referente à cor desejada por ele, e então, por meio da liberação de pigmentos e mistura da tinta, deverá manipular uma lata de tinta inicialmente branca a fim de transformá-la em uma lata de tinta da cor desejada. \par
            \begin{figure}[h]
                \centering
                \includegraphics[width=10cm]{diagramaCores.jpg}
                \caption{Figura 1: Diferença entre CMYK e RGB.}
                \label{fig:cores}
            \end{figure}
            O código recebido deve estar de acordo com a escala CMYK (Cyan - ciano, Magenta, Yellow - amarelo, Key - preto). Essa escala é um sistema subtrativo, sendo assim, os pigmentos são adicionados de maneira a remover cor até chegar ao tom pretendido. Dessa forma, os pigmentos servem como filtro, absorvendo determinados comprimentos de onda e manipulando qual cor será visualizada. A diferença entre o CMYK e o RGB se dá pois o primeiro é composto por cor pigmento enquanto o segundo é composto por cor luz.\par
            Dessa forma, para o CMYK, a injeção de cores da máquina será feita por 4 válvulas, uma para cada pigmento, de forma a injetar as cores na base branca da lata. Será considerado que 1 gota de pigmento equivale a 0,05ml, e são necessários 100 milissegundos com a válvula aberta para que 1 gota caia.\par
            O primeiro passo de funcionamento, é um estado de início. Nesse estado, os registradores são zerados, assim como os contadores. Isso evita que alguma informação de "lixo" entre no processo que será iniciado, garantindo maior segurança de que o resultado será o desejado.\par
            O segundo passo, é um estado de espera. Nele, a máquina está aguardando que o usuário entre com um código válido de cor CMYK e posicione a lata no local correto. O código é chamado de "Codigo\textunderscore cor" e é uma entrada de 32 bits, enquanto a posição da lata será tratada como uma entrada de um único bit.\par
            A partir disso, a máquina avança para um estado de validação. Nesse estado, o código inserido será validado, ou seja, o programa deve retornar se o que o usuário forneceu como código de cor é de fato um código válido. Isso resulta em uma bifurcação no diagrama, já que, dependendo se o código for válido ou não, a tratativa sobre ele será diferente. Se for inválido, a máquina passa para o estado de erro. No caso do código ser válido, o próximo passo é o estado de pigmentação.\par
            No caso de erro, o próximo estado passa a ser o início, a partir de um reset. Esse estado emite um código de erro para que o usuário possa recomeçar o processo depois de corrigir o que havia de errado da primeira vez.\par
            No caso onde o código é válido e o processo segue para a pigmentação, Nesse estado, será calculado quanto tempo cada válvula deve ficar aberta para que a tinta resultante tenha a cor desejada. Após isso, as válvulas são energizadas, o que nesse trabalho será representado por leds acesos.\par
            Após a adição de pigmento, o processo entra no estado de mistura. A partir da agitação da lata, os pigmentos são misturados na tinta. Esse processo também será representado com um led aceso.\par
            Seguinte à mistura, a máquina deve verificar se o resultado obtido do processo é igual ao resultado desejado pelo usuário, sendo assim, inicia-se o estado de verificação. Na prática, isso seria feito a partir de um sensor de cor e a comparação do código de entrada e o código retornado pelo sensor. Nesse trabalho, usaremos apenas para teste com códigos inseridos manualmente. Caso os códigos batam, a máquina segue para o estado de pronto. No caso de ter sido obtido um resultado indesejado, inicia-se o estado de erro, que dá a mesma tratativa já descrita após o processo de validação.\par
            No último estágio do ciclo, tem-se a tinta pronta. Esse estado aguarda que a lata de tinta seja retirada do local reservado, para que o ciclo possa ser reiniciado.\par
            
        \section{Componentes}
            \subsection{Registrador}
                O registrador possui uma entrada load, que habilita que o dado seja salvo, uma entrada de reset, uma entrada de dados e uma saída de dados. O tamanho da entrada e da saída foi definido como "generic" para possibilitar o uso do componente em instâncias com tamanhos diferentes.

                \begin{figure}[h!]
                    \centering
                    \includegraphics[width = .9 \textwidth]{Reg/Reg.png}
                    \caption{Código VHDL do registrador}
                    \label{fig:Reg}
                \end{figure}

                \begin{figure}[h!]
                    \centering
                    \includegraphics[width = .9 \textwidth]{Reg/tb_Reg.png}
                    \caption{Código VHDL do testbench do registrador}
                    \label{fig:tb_Reg}
                \end{figure}

                \begin{figure}[h!]
                    \centering
                    \includegraphics[width = .9 \textwidth]{Reg/sim_Reg.png}
                    \caption{Resultado da simulação do registrador}
                    \label{fig:sim_Reg}
                \end{figure}
                
            \subsection{Comparador}
                O projeto do comparador, possui duas entradas de dados e três saídas, igual, maior e menor. As saídas são mantidas em nível lógico baixo e somente são alteradas caso o resultado da respectiva comparação seja verdadeiro. Do mesmo modo do registrador, o tamanhos da entrada de dados foi definido como "generic".

                \begin{figure}[h!]
                    \centering
                    \includegraphics[width = .9 \textwidth]{Comp/Comp.png}
                    \caption{Código VHDL do comparador}
                    \label{fig:Comp}
                \end{figure}

                \begin{figure}[h!]
                    \centering
                    \includegraphics[width = .9 \textwidth]{Comp/tb_Comp.png}
                    \caption{Código VHDL do testbench do comparador}
                    \label{fig:tb_Comp}
                \end{figure}

                \begin{figure}[h!]
                    \centering
                    \includegraphics[width = .9 \textwidth]{Comp/sim_Comp.png}
                    \caption{Resultado da simulação do comparador}
                    \label{fig:sim_Comp}
                \end{figure}
            \subsection{CodValido}
                Este componente é o responsável por verificar a validade do código. Possui uma entrada de 32 bits referente ao código da cor e uma saída que indica a validade. O código é no formato da escala CMYK, desse modo cada posição corresponde a uma cor e deve estar no intervalo de $0\approx100$, em decimal. Como a entrada do código é em hexadecimal, o valor deve ir de $0\approx64$. O componente então, verifica se a entrada encontra-se nesse intervalo.

 \chapter{CONCLUSÃO}
	
	\begin{thebibliography}{1}
		%\bibitem{ref1}  $\ll$ endereço do site $\gg$
            \bibitem{ref1}  $\ll$ https://www.printi.com.br/blog/cores-cmyk-no-sistema-de-impressao $\gg$
            \bibitem{ref2}  $\ll$ https://tecnoblog.net/responde/o-que-sao-os-padroes-de-cores-rgb-e-cmyk/ $\gg$
            \bibitem{ref3} $\ll$ https://www.people.com.br/noticias/design/qual-a-diferenca-entre-cmyk-e-rgb $\gg$
	\end{thebibliography}

\end{document}